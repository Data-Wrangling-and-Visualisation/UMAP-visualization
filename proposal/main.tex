\documentclass{article}
\usepackage[utf8]{inputenc}
\usepackage{geometry}
\geometry{a4paper, margin=1in}
\usepackage{graphicx}
\usepackage{hyperref}
\usepackage{datetime}
\hypersetup{
	colorlinks=true, 
	linkcolor=blue, 
	urlcolor=blue, 
	citecolor=blue
}

\setlength{\parindent}{0pt} % Disable indentation for new paragraphs

\title{Data Wrangling \& Visualization: Project Proposal \\
 \textbf{Interactive Visualization of \texttt{UMAP} Algorithm on User Data}}
\author{Nikita Zagainov, Dmitry Tetkin, Nikita Tsukanov}
\date{\monthname[\the\month] \the\year}

\begin{document}

\maketitle

\section{Project Vision and Goal}
This project aims to create an interactive website for explaining
one of the most powerful dimensionality reduction algorithms,
\texttt{UMAP} (Uniform Manifold Approximation and Projection).

\medskip

We aim to provide a comprehensive explanation of the algorithm, and
demonstrate its application to table data, which can be chosen by the
user

\section{Data}
The data for our project is arbitrarily chosen by the user, a few
restriction apply to the data:
\begin{itemize}
	\item The data should be in a table format
	\item The data should be in a \texttt{.csv} format
	\item The data should contain at least 2 columns
	      (for \texttt{2D} visualization) or 3 columns
	      (for \texttt{3D} visualization)
	\item All columns should be numerical or categorical
	      (arbitrary data type). Irrelevant columns will be ignored
	\item Optionally, the user can provide a label column in form of
	      additional \texttt{.csv} file, which will be used for coloring
	      the data points
\end{itemize}

For example purposes, we will provide \texttt{MNIST} (can be found
\href{https://pytorch.org/vision/main/generated/torchvision.datasets.MNIST.html}{here})
dataset in
\texttt{.csv} format, and visualization of this data.

\section{App Architecture \& Pipeline}

We omit data scrapping step, as it is not required for this project.
All other obligatory steps are provided:
\begin{enumerate}
	\item \textbf{Data Cleaning and Preprocessing:} to make sure that
	      our engine works with the data, first step of our pipeline cleans
	      and preprocesses the input data, and performs necessary checks
	\item \textbf{UMAP Algorithm:} this step applies \texttt{UMAP}
	      algorithm to the input data and streams the resulting
	      embeddings on each iteration to the visualization app. For
	      algorithm implementation, we will use original \texttt{UMAP}
	      \href{https://github.com/lmcinnes/umap/}{implementation}
	      as backbone with some modifications (control over NN search
	      algorithm, embedding streaming) in \texttt{NumPy},
	      \texttt{SciPy}
	\item \textbf{Data Delivery:} for streaming purposes, we will use
	      \texttt{REST API} to deliver the embeddings to the frontend.
	      For implementation, we will use \texttt{FastAPI} library
	\item \textbf{Visualization App:} the embeddings are streamed to the
	      visualization app, which will display the process in real time
	      in 2 or 3 dimensions, allowing user to interact with the
	      visualization (zoom, pan, rotate, etc.). Fronted app will use
	      \texttt{D3.js} library for visualization
\end{enumerate}

\section{Visualization Features}
Proposed features of the app:
\begin{itemize}
	\item We aim to provide as much interactivity as possible, allowing
	      the user to insert their own data in \texttt{.csv} format
	\item Optionally, the user can provide a label column in form of
	      additional \texttt{.csv} file, which will be used for coloring
	      the data points
	\item For better understanding of the algorithm, we allow user to
	      set desired parameters of the \texttt{UMAP} algorithm which will
	      be applied to the data
	      (2 or 3 dimensions, number of neighbors, etc.)
	\item Real-time process of the algorithm will be displayed
	      on the page, with ability to pause, rewind, fast-forward, and
	      navigate through the embedding space
\end{itemize}

\section{Project Timeline and Milestones}
We cannot be certain about the exact timeline of the project, but we
provide a rough estimation of the project milestones:

\begin{itemize}
	\item \textbf{Week 1-2:} Project Setup, gathering with a team,
	      finding hosting service, creating basic frontend page
	\item \textbf{Week 3:} \texttt{UMAP} algorithm implementation,
	      possibly some optimizations and testing
	\item \textbf{Week 4-5:} Creating pipeline for effective data delivery
	\item \textbf{Week 6-7:} Creating frontend visualization app
	\item \textbf{Week 8:} Final testing and deployment, presentation
	      preparation
\end{itemize}

\end{document}